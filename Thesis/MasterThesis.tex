\documentclass[twoside,english]{uiofysmaster/uiofysmaster}

\usepackage[toc,titletoc,title,page]{appendix} %to add appendices (and have them in toc)
\usepackage[utf8]{inputenc}
%\usepackage{mhchem} %latex chemistry symbols
\usepackage{blindtext} %to fill in dummy text
%\usepackage{cite} %to have multiple citations in one \cite{key1,key2,..} -do not use with natbib!!
\usepackage{tcolorbox} %to have boxes w color around text and math mode
\usepackage{enumitem} %to reduce vertical spacing in enumerate
\usepackage{tabu} % to set tables to page width
%\usepackage{aas_macros}

\usepackage[sort&compress,square,comma,numbers]{natbib} %to use \citet, now mixed with [nr]
\usepackage[nottoc]{tocbibind}

\usepackage{float} 

\hypersetup{
    colorlinks = true,
    citecolor = blue
}

\interfootnotelinepenalty=10000 % to force footnotes to NOT run over to the next page

%---
% to reduce space around table of contents (to fit everything into one page): 
\usepackage{tocloft}
\setlength{\cftbeforetoctitleskip}{0pt}
\setlength{\cftaftertoctitleskip}{0pt}
%---

\usepackage{epigraph}
\setlength\epigraphwidth{11cm}
\setlength\epigraphrule{0pt}

%---
\newcommand{\Sm}{$^{140}$Sm} % making it faster to write Sm140
\newcommand{\Pb}{$^{208}$Pb} 
\newcommand{\bd}{$\beta$-decay} % making it faster to write 
\newcommand{\ga}{$\gamma$}

%---
% modifying color in code listings and some style
\usepackage{color}
 
%\definecolor{codegreen}{rgb}{0,0.6,0} % too flashy
\definecolor{codegreen}{rgb}{0.0, 0.42, 0.24} % less flashy so comments not take all attention
\definecolor{codegray}{rgb}{0.5,0.5,0.5}
\definecolor{codepurple}{rgb}{0.58,0,0.82}
%\definecolor{codepurple}{rgb}{1.0, 0.0, 0.22} %carminered, could try it 
%\definecolor{backcolour}{rgb}{0.95,0.95,0.92} % original suggestion
\definecolor{backcolour}{rgb}{0.94, 0.97, 1.0}% aliceblue, not so flashy and not as ugly
 
\lstdefinestyle{mystyle}{
    backgroundcolor=\color{backcolour},   
    commentstyle=\color{codegreen},
    %commentstyle=\color{codegray},    
    keywordstyle=\color{magenta},
    numberstyle=\tiny\color{codegray},
    stringstyle=\color{codepurple},
    basicstyle=\footnotesize,
    breakatwhitespace=false,         
    breaklines=true,                 
    captionpos=b,                    
    keepspaces=true,                 
    %numbers=left,     %removing line numbers in the code snippets               
    %numbersep=5pt,                  
    showspaces=false,                
    showstringspaces=false,
    showtabs=false,                  
    tabsize=2,
    %float=tp,
    %floatplacement=tbp
}
 
\lstset{style=mystyle}
\renewcommand{\lstlistingname}{Code}
%---

%---
% new tcolorbox environment
% #1: tcolorbox options
% #2: color
% #3: box title
\newtcolorbox{mybox}[3][]
{
  colframe = #2!25,
  colback  = #2!10,
  coltitle = #2!20!black,  
  title    = #3,
  #1,
}

%---


%\bibliography{references}

\author{Trond Wiggo Johansen}
\title{I did something cool at CERN - ISOLDE
}
\date{May 2019}
 
% ----------------------------------------------------------------------------------------------------------------------
% ----------------------------------------------------------------------------------------------------------------------
%Equations
%
%The command \eqref{} works exactly like \ref{}, but it adds parantheses to a plain number.
%
%Figures and tables
%
%\autoref{} is a usefull command when refering to to figures and tables. The command creates a reference with additional text corresponding to the target's type. For example, the command \autoref{fig:myfigure} would create a hyperlink to the \label{fig:myfigure} command, wherever it is. Assuming that this label is pointing to a figure, the hyperlink would contain the text "Figure 1.1", or similar.

%Two basic citation commands, \citet and \citep for textual and parenthetical citations, respectively. …
%\citet{jon90} --> Jones et al. (1990)
%\citep{jon90} --> (Jones et al., 1990)
%\citet*{jon90} --> Jones, Baker, and Williams (1990)
%\citep*{jon90} --> (Jones, Baker, and Williams, 1990)


\begin{document}

% set space around equations
\setlength{\belowdisplayskip}{12pt} \setlength{\belowdisplayshortskip}{12pt}
\setlength{\abovedisplayskip}{12pt} \setlength{\abovedisplayshortskip}{12pt}

\maketitle

%Centering the front page, see: https://github.com/ComputationalPhysics/uiofysmaster

%%% ABSTRACT
\begin{abstract}


\end{abstract}


\begin{dedication}
To my family, for all their support and encouragement!

\end{dedication}

\begin{acknowledgements}
Supervisors Andreas Görgen and Katarzyna Hady\'nska-Kl\c ek

Nuclear Physics Group

Computational Physics Group, Morten Hjorth-Jensen

CERN-ISOLDE, Liam Gaffney

Lillefy, FFU, Fysikkforeningen

My family

Morten, Alex and Astrid

Ina, I love you.

\subsection*{Collaboration details}
The sorting and analysis code used in this thesis has been developed at CERN-ISOLDE and can be found at \url{https://github.com/Miniball/MiniballCoulexSort}

The code for theoretical predictions of energy used in the calibration was developed by Liam Gaffney who is working at ISOLDE and has to do with analysis of data from Miniball and ISS. kinsim can be found here \url{https://github.com/lpgaff/kinsim}

Some calibration code is based on the codes of Ville Virtanen and Liam Gaffney. 

Other code/scripts have been written by the author. C++ / Python.

  \vspace{1.5cm}
  
  \noindent\textit{Trond Wiggo Johansen}\\
  
  \noindent September, 2019
  
\end{acknowledgements}


\tableofcontents


% ----------------------------------------------------------------------------------------------------------------------
% ----------------------------------------------------------------------------------------------------------------------

\chapter{Introduction}
Test \cite{Clement2016}

Test 2 \cite{OCLweb}

kinsim \cite{kinsim}

The experiment has been done before, with lower energy (and another target), Malin Klintefjord. \url{http://urn.nb.no/URN:NBN:no-56121} \newline
 
 
Experiment conducted 8th - 14th of August 2017.
% ----------------------------------------------------------------------------------------------------------------------
% ----------------------------------------------------------------------------------------------------------------------

\chapter{Theory?}

Deformation. \newline

Shape coexistence?

% ----------------------------------------------------------------------------------------------------------------------
% ----------------------------------------------------------------------------------------------------------------------


\chapter{Coulomb excitation experiment} 

\begin{table}[H]
  \centering
  \caption{Acronyms and abbreviations.}
    \begin{tabular}{ll}
        \hline
        PSB & Proton Synchrotron Booster \\
        ISOLDE & Isotope Separator On-Line DEvice \\
        GPS & General Purpose Separator \\
        HRS & High Resolution Separator \\
        EBIS & Electron Beam Ion Source \\
        REXEBIS & Radioactive beam EXperiment EBIS \\
        RILIS & Resonance Ionization Laser Ion Source \\
        HIE-ISOLDE & High Intensity and Energy upgrade \\
        RIB & Radioactive Ion Beam \\
        ENSAR2 & European Nuclear Science and Applications Research - 2 \\
        ISOL & Isotope Separator On-Line \\
        Linac & Linear accelerator \\
        ADC & Analog-to-Digital Converter \\
        TDC & Time-to-Digital Converter (or time digitizer) \\
        Coulex & Coulomb excitation \\
        \hline
    \end{tabular}
    \label{tab:acro}
\end{table}


\textcolor{red}{---------} \newline
\textcolor{red}{Oppgavens mål:} \newline
The ISOLDE facility at CERN has been upgraded to provide higher energies and intensities for radioactive ion beams. A new experiment to study 140Sm was performed in the summer of 2017. The goal of the experiment was to measure electromagnetic transition probabilities and electric quadrupole moments for several excited states in 140Sm by measuring Coulomb excitation probabilities. A large data set was obtained using silicon detectors to determine the energies and angles of scattered particles, and germanium detectors to measure gamma rays from excited states in 140Sm. \newline

The goal of the master thesis is to analyze the data from this experiment. The required tasks include development and improvement of data analysis software to determine Coulomb excitation yields. These yields will then, in a second step, be compared to theoretical calculations and transition probabilities and quadrupole moments will be extracted using chi-square minimization procedures. \newline


\textcolor{red}{Prosjektbeskrivelse (omfang 60 studiepoeng):} \newline
The shape of an atomic nucleus is determined by a delicate interplay between macroscopic (liquid drop) properties and microscopic shell effects. Nuclei with filled proton or neutron shells (i.e. magic nuclei) are generally spherical in shape, whereas nuclei with open shells gain energy by assuming a deformed shape. Depending on the occupation of specific orbitals, the nuclear shape can change drastically by adding or removing protons or neutrons. Certain nuclei exhibit shape coexistence, i.e. the coexistence of quantum states that correspond to different shapes. Because the shape of a nucleus is so sensitive to the underlying nuclear structure and to changes of the proton and neutron numbers, the excitation energy, or the angular momentum, observables related to the nuclear shape are used as benchmarks for theoretical models. 

Nuclei in the rare earth region, and in particular the chain of samarium isotopes, exhibit a variety of shape effects. The Sm isotope with closed neutron shell at N=82, 144Sm, is spherical in shape. Adding neutrons to 144Sm changes the deformation to an elongated (prolate) quadrupole shape. The transition from spherical to prolate shape, which occurs for 152Sm at N=90, can be interpreted as a shape-phase transition. Flattened (oblate) quadrupole shapes are predicted by theory to occur below the N=82 shell closure. An earlier experiment studying 140Sm at CERN-ISOLDE found triaxial shape for this isotope, i.e. a shape where all three principal axes of the ellipsoid have different lengths. 140Sm can therefore be considered to lie at the critical point of a phase transition from spherical to deformed, and from prolate to oblate shape. \newline

\textcolor{red}{Foreløpig tittel:} \newline
Coulomb excitation of 140Sm \newline


\textcolor{red}{Metoder som tenkes benyttet:} \newline
Multi-step Coulomb excitation with radioactive beam, isotope separation on-line technique, nuclear spectroscopy, particle-gamma and particle gamma-gamma coincidence analysis, advanced chi-square minimization procedures. \newline
\textcolor{red}{---------} \newline


\section{ISOLDE}

The ISOLDE Radioactive Ion Beam facility \newline

Nuclear physics facility at CERN.


ISOLDE \url{http://isolde.web.cern.ch}

REX-ISOLDE \url{http://rex-isolde.web.cern.ch}

RILIS \url{http://rilis.web.cern.ch}

HIE-ISOLDE \url{http://hie-isolde-project.web.cern.ch}, technical design \url{http://cds.cern.ch/record/2635892?ln=en}, direct to doc: \url{http://cds.cern.ch/record/2635892/files/HIE-ISOLDE_TDR.pdf}

MINIBALL \url{http://isolde.web.cern.ch/experiments/miniball} and \url{https://www.miniball.york.ac.uk/wiki/Main_Page}

ENSAR2 \url{http://www.ensarfp7.eu}


Test \cite{CERN-AC}, copyright: \url{https://copyright.web.cern.ch}

CERN Document Server  \url{https://cds.cern.ch}

\subsection{Beam production}

ISOLDE experimental hall:
\begin{align*}
	\text{p}^+ \rightarrow~ &\text{Ta} \rightarrow (\text{Produces the chart of nuclides up to Ta}) \rightarrow \text{GPS} \rightarrow \\
	&\text{RILIS} \rightarrow \text{REXEBIS} \rightarrow \text{HIE-ISOLDE} \rightarrow \text{MINIBALL}
\end{align*}


Protons from the PSB comes into the ISOLDE facility and hit a production target of tantalum, producing the elements in the chart of nuclides up to tantalum. The fragments travels onward to the GPS where the mass of $A = 140$ is selected. RILIS selects samarium with laser. REXEBIS excites the nucleus in three steps ionizing the atom, which leaves the nucleus in a high charge state. The HIE-linac accelerates the beam through the beam line and magnets bend the beam into MINIBALL. 


Magnets....

\bigskip

Ebis: charge breader: release beam with certain energy.



high-performance charge breeder (CB). CB based on the Electron Beam Ion Source (EBIS) technology – an EBIS Charge Breeder (ECB)


\bigskip

\textcolor{red}{HRS not in use for this experiment?}


HIE-ISOLDE (Superconducting Linac Upgrade): Linear accelerator, HIE-linac \newline

Post-accelerated beams ISOLDE \url{http://iopscience.iop.org/article/10.1088/1361-6471/aa78ca}


ISOLDE actually uses the most protons at CERN [\textcolor{red}{ref?}].


Very pure beam

\subsection{Target}

\Pb\ was chosen as a target. Want high $Z$ so that the probability of excitation is high. Not enough beam energy to excite \Pb. \newline


Contamination... finger print [\textcolor{red}{picture}]


\subsection{Miniball}

Pictures \url{https://cds.cern.ch/record/844871?ln=en}

\subsection{Particle detector, DSSSD}

16 rings, 12 strips effectively (24 strips, 12 pairs with two strips making a pair)


\subsection{\ga\ detectors, HPGe}

24 six-fold segmented

\bigskip

Cryo-modules

\section{Experimental setup}
\Sm ~Coulomb excitation experiment.

Experiment code: IS558 

Ta: tantalum (Z = 73)

Sm: samarium (Z = 62)

Pb: lead (Z = 82) \newline



Beam: \Sm (4.65 MeV/$u$, 651 MeV)

Target: \Pb


Small angle: Forward scattering: Larger distance, weaker \textcolor{red}{EM}-field, less excitation probability.

Large angle: Backward scattering: Closer distance, stronger \textcolor{red}{EM}-field, higher excitation probability. \newline






% ----------------------------------------------------------------------------------------------------------------------% ----------------------------------------------------------------------------------------------------------------------



\chapter{Data analysis} 

ROOT: analysere data

kinsim3 + SRIM

 +Experiment code: IS553

Ni: nickel (Z = 28)

Ba: barium (Z = 56)


\bigskip

particle-gamma and particle-gamma-gamma coincidence

\section{Calibration}
\subsection{Particle detector}
ADC: Analog to digital converter (Mesytec)

TDC: Time to digital converter

DSSSD: Double-Sided Silicon Strip Detector $\implies$ CD

must remove the inner ring from data analysis because of damage

\begin{align*}
	gain = \frac{E_{Sm} - E_{Pb ~or~ Ni}}{Ch_{Sm} - Ch_{Pb ~or~ Ni}}
\end{align*}

\begin{align*}
	offset = E_{Sm} - gain \cdot Ch_{Sm}
\end{align*}
in keV.



\subsection{Gamma detectors}

DGF: Digital \ga~ finder

\section{Doppler correction}

% ----------------------------------------------------------------------------------------------------------------------% ----------------------------------------------------------------------------------------------------------------------


\chapter{Experimental results}


% ----------------------------------------------------------------------------------------------------------------------% ----------------------------------------------------------------------------------------------------------------------


\chapter{Discussion}


% ----------------------------------------------------------------------------------------------------------------------% ----------------------------------------------------------------------------------------------------------------------

\chapter{Summary and outlook}


% ----------------------------------------------------------------------------------------------------------------------% ----------------------------------------------------------------------------------------------------------------------




% ----------------------------------------------------------------------------------------------------------------------% ----------------------------------------------------------------------------------------------------------------------

\begin{appendices}
\chapter{Some Appendix}


\chapter{Some other appendix...}

\end{appendices}



%\bibliographystyle{unsrtnat}
\bibliographystyle{mybibstyle} %my own bibstyle to set first names to one letter + unsrtnat

\bibliography{/Users/trondwj/GitHub/MasterThesis/Thesis/Mendeley/test.bib,/Users/trondwj/GitHub/MasterThesis/Thesis/References/web_references.bib}


\end{document}