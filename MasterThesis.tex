\documentclass[twoside,english]{uiofysmaster/uiofysmaster}

\usepackage[toc,titletoc,title,page]{appendix} %to add appendices (and have them in toc)
\usepackage[utf8]{inputenc}
%\usepackage{mhchem} %latex chemistry symbols
\usepackage{blindtext} %to fill in dummy text
%\usepackage{cite} %to have multiple citations in one \cite{key1,key2,..} -do not use with natbib!!
\usepackage{tcolorbox} %to have boxes w color around text and math mode
\usepackage{enumitem} %to reduce vertical spacing in enumerate
\usepackage{tabu} % to set tables to page width
%\usepackage{aas_macros}

\usepackage[sort&compress,square,comma,numbers]{natbib} %to use \citet, now mixed with [nr]
\usepackage[nottoc]{tocbibind}

\interfootnotelinepenalty=10000 % to force footnotes to NOT run over to the next page

%---
% to reduce space around table of contents (to fit everything into one page): 
\usepackage{tocloft}
\setlength{\cftbeforetoctitleskip}{0pt}
\setlength{\cftaftertoctitleskip}{0pt}
%---

\usepackage{epigraph}
\setlength\epigraphwidth{11cm}
\setlength\epigraphrule{0pt}

%---
\newcommand{\Sm}{$^{140}$Sm} % making it faster to write Sm140
%\newcommand{\Osreac}{${}^{192}$Os$(\alpha$, $\alpha ' \gamma){}^{192}$Os}
%\newcommand{\Osreacto}{${}^{191}$Os$(n$, $\gamma){}^{192}$Os}
\newcommand{\bd}{$\beta$-decay} % making it faster to write 
%\newcommand{\bref}{B$^2$FH} % making it faster to write 
%\newcommand{\gsf}{$\gamma$-strength function}
\newcommand{\ga}{$\gamma$}

%---
% modifying color in code listings and some style
\usepackage{color}
 
%\definecolor{codegreen}{rgb}{0,0.6,0} % too flashy
\definecolor{codegreen}{rgb}{0.0, 0.42, 0.24} % less flashy so comments not take all attention
\definecolor{codegray}{rgb}{0.5,0.5,0.5}
\definecolor{codepurple}{rgb}{0.58,0,0.82}
%\definecolor{codepurple}{rgb}{1.0, 0.0, 0.22} %carminered, could try it 
%\definecolor{backcolour}{rgb}{0.95,0.95,0.92} % original suggestion
\definecolor{backcolour}{rgb}{0.94, 0.97, 1.0}% aliceblue, not so flashy and not as ugly
 
\lstdefinestyle{mystyle}{
    backgroundcolor=\color{backcolour},   
    commentstyle=\color{codegreen},
    %commentstyle=\color{codegray},    
    keywordstyle=\color{magenta},
    numberstyle=\tiny\color{codegray},
    stringstyle=\color{codepurple},
    basicstyle=\footnotesize,
    breakatwhitespace=false,         
    breaklines=true,                 
    captionpos=b,                    
    keepspaces=true,                 
    %numbers=left,     %removing line numbers in the code snippets               
    %numbersep=5pt,                  
    showspaces=false,                
    showstringspaces=false,
    showtabs=false,                  
    tabsize=2,
    %float=tp,
    %floatplacement=tbp
}
 
\lstset{style=mystyle}
\renewcommand{\lstlistingname}{Code}
%---

%---
% new tcolorbox environment
% #1: tcolorbox options
% #2: color
% #3: box title
\newtcolorbox{mybox}[3][]
{
  colframe = #2!25,
  colback  = #2!10,
  coltitle = #2!20!black,  
  title    = #3,
  #1,
}

%---


%\bibliography{references}

\author{Trond Wiggo Johansen}
\title{I did something cool at CERN - ISOLDE
}
\date{May 2019}
 
% ----------------------------------------------------------------------------------------------------------------------
% ----------------------------------------------------------------------------------------------------------------------
%Equations
%
%The command \eqref{} works exactly like \ref{}, but it adds parantheses to a plain number.
%
%Figures and tables
%
%\autoref{} is a usefull command when refering to to figures and tables. The command creates a reference with additional text corresponding to the target's type. For example, the command \autoref{fig:myfigure} would create a hyperlink to the \label{fig:myfigure} command, wherever it is. Assuming that this label is pointing to a figure, the hyperlink would contain the text "Figure 1.1", or similar.

%Two basic citation commands, \citet and \citep for textual and parenthetical citations, respectively. …
%\citet{jon90} --> Jones et al. (1990)
%\citep{jon90} --> (Jones et al., 1990)
%\citet*{jon90} --> Jones, Baker, and Williams (1990)
%\citep*{jon90} --> (Jones, Baker, and Williams, 1990)


\begin{document}

% set space around equations
\setlength{\belowdisplayskip}{12pt} \setlength{\belowdisplayshortskip}{12pt}
\setlength{\abovedisplayskip}{12pt} \setlength{\abovedisplayshortskip}{12pt}

\maketitle

%Centering the front page, see: https://github.com/ComputationalPhysics/uiofysmaster

%%% ABSTRACT
\begin{abstract}


\end{abstract}


\begin{dedication}
To my family, for all their support and encouragement!

\end{dedication}

\begin{acknowledgements}
Supervisors Andreas Görgen and Katarzyna Hady\'nska-Kl\c ek

Nuclear Physics Group

Computational Physics Group, Morten Hjorth-Jensen

CERN-ISOLDE, Liam Gaffney

Lillefy, FFU, Fysikkforeningen

My family

Morten, Alex and Astrid

Ina, I love you.

\subsection*{Collaboration details}
The sorting and analysis code used in this thesis has been developed at CERN-ISOLDE and can be found at \url{https://github.com/Miniball/MiniballCoulexSort}

Other code/scripts have been written by the author. C++ / Python.

  \vspace{1.5cm}
  
  \noindent\textit{Trond Wiggo Johansen}\\
  
  \noindent September, 2019
  
\end{acknowledgements}


\tableofcontents


% ----------------------------------------------------------------------------------------------------------------------
% ----------------------------------------------------------------------------------------------------------------------

\chapter{Introduction}
Test \cite{Clement2016}

The experiment has been done before, with lower energy (and another target), Malin Klintefjord.
 
% ----------------------------------------------------------------------------------------------------------------------
% ----------------------------------------------------------------------------------------------------------------------

\chapter{Theory?}

% ----------------------------------------------------------------------------------------------------------------------
% ----------------------------------------------------------------------------------------------------------------------


\chapter{Coulomb excitation experiment} 
\section{ISOLDE}
\subsection{Miniball}
\subsection{DSSSD}

\section{Experimental setup}
\Sm ~Coulomb excitation experiment.

% ----------------------------------------------------------------------------------------------------------------------% ----------------------------------------------------------------------------------------------------------------------



\chapter{Data analysis} 
\section{Calibration}
\subsection{Particle detector}
DSSSD: Double-Sided Silicon Strip Detector $\implies$ CD

\subsection{Gamma detectors}

\section{Doppler correction}

% ----------------------------------------------------------------------------------------------------------------------% ----------------------------------------------------------------------------------------------------------------------


\chapter{Experimental results}


% ----------------------------------------------------------------------------------------------------------------------% ----------------------------------------------------------------------------------------------------------------------


\chapter{Discussion}


% ----------------------------------------------------------------------------------------------------------------------% ----------------------------------------------------------------------------------------------------------------------

\chapter{Summary and outlook}


% ----------------------------------------------------------------------------------------------------------------------% ----------------------------------------------------------------------------------------------------------------------




% ----------------------------------------------------------------------------------------------------------------------% ----------------------------------------------------------------------------------------------------------------------

\begin{appendices}
\chapter{Some Appendix}


\chapter{Some other appendix...}

\end{appendices}



%\bibliographystyle{unsrtnat}
\bibliographystyle{mybibstyle} %my own bibstyle to set first names to one letter + unsrtnat

\bibliography{/Users/trondwj/Documents/Mendeley/test.bib}


\end{document}